% That's an example of XeLaTeX document we'd like to be able to create

\documentclass[10pt]{article}
\usepackage[a4paper,margin=1in]{geometry} % set smaller margins %
\usepackage{amssymb} % use standard math symbols %
\usepackage{amsmath} % use \text inside formulas and for \eqref %
\usepackage{xltxtra} % enable full XeLaTeX power %

%\defaultfontfeatures{}
\setmainfont{Georgia}

% set paragraph margins %
\setlength{\parindent}{0pt}
\setlength{\parskip}{6pt plus 2pt minus 1pt}
\setlength{\emergencystretch}{3em} % prevent overfull lines

\usepackage{graphicx} % for \includegraphics
\usepackage{float} % for exact placement of figures

% setup hyperlinks support
\usepackage[colorlinks,urlcolor=blue,filecolor=blue,linkcolor=black,citecolor=black]{hyperref}
\usepackage[all]{hypcap}
\urlstyle{same}

\begin{document}

\section{Introduction}

This document shows what should be possible with
\href{http://textjs.org/}{\texttt{texts.js}} and
\href{http://en.wikipedia.org/wiki/XeTeX}{\XeTeX{}}.

\section{Headings}

\subsection{Subsection Heading}

\subsubsection{Subsubsection Heading}
\label{sec:somesection}

\section{Text Formatting}

You can use \emph{emphasis} and \textbf{strong} emphasis. Inline \texttt{code}
will use monspaced font.

\begin{verbatim}
Code blocks use monospaced font as well and preserve line
breaks
\end{verbatim}

\section{Math}

Math formulas can be used inside paragraph like $E=mc^2$ or on a separate line
like the following one.

\begin{equation}
\label{eq:myequation}
1+\frac{q^2}{(1-q)}+\frac{q^6}{(1-q)(1-q^2)}+\cdots =
\prod_{j=0}^{\infty}\frac{1}{(1-q^{5j+2})(1-q^{5j+3})},
\quad\quad \text{for $|q|<1$}
\end{equation}

\section{Footnotes}

Footnotes are placed inside text\footnote{Yes right here.} in the \TeX{} source
file but appear at the bottom of the page.

Or you could reference\footnotemark{} your footnote or two\footnotemark{}.

\footnotetext{And then define it.}

\footnotetext{I mean, both of them.}

\section{Cross-references}

We can refer to Section~\ref{sec:somesection},
Equation~\eqref{eq:myequation}, Figure~\ref{fig:logo} or
Table~\ref{tab:example}.

\section{Hyperlinks}

Cross-references are hyperlinked automatically (try clicking the figure number
above). It is possible to include surrounding text in the link, e.g.
\hyperref[fig:logo]{Figure~\ref*{fig:logo}}—so that
link is larger.

And you can add arbitrary hyperlinks of course.
Link \href{http://www.texts.io/}{text} can differ from it's URL or be the same
as the URL: \url{http://www.google.com/}.

You can include e-mail links:
\href{mailto:sheremetyev@gmail.com}{sheremetyev@gmail.com}. Local files can be
\href{run:basic.pdf}{referenced}. It is possible to
\hyperlink{mylabel}{reference} any location in the document.

\hypertarget{mylabel}{}
For example, this text is referenced in previous paragraph.

\section{Images}

\begin{figure}[H]
    \centering
    \includegraphics{Texts_Logo.png}
    \caption{Texts editor logo}
    \label{fig:logo}
\end{figure}

\section{Tables}

\begin{table}[H]
\centering
\begin{tabular}{ | l | l | l | p{5cm} |}
\hline
Day & Min Temp & Max Temp & Summary \\ \hline
Monday & 11C & 22C & A clear day with lots of sunshine.  However, the strong breeze will bring down the temperatures. \\ \hline
Tuesday & 9C & 19C & Cloudy with rain, across many northern regions. Clear spells across most of Scotland and Northern Ireland, but rain reaching the far northwest. \\ \hline
Wednesday & 10C & 21C & Rain will still linger for the morning. Conditions will improve by early afternoon and continue throughout the evening. \\
\hline
\end{tabular}
\caption{Simple table}
\label{tab:example}
\end{table}

\section{Bibliography}

You define bibliographic sources at the bottom of the file and
use~\cite{lamport94,goossens93} them anywhere in the text.

\section{Lists}

\begin{itemize}

\item First bulleted item.

\item Second bulleted item.

    \begin{itemize}

    \item Subitem.

        \begin{itemize}

        \item Subsubitem.

        \end{itemize}

    \end{itemize}

\end{itemize}


\begin{enumerate}

\item First numbered item.

\item Second numbered item.

    \begin{enumerate}

    \item Subitem.

        \begin{enumerate}

        \item Subsubitem.

        \end{enumerate}

    \end{enumerate}

\end{enumerate}

\begin{description}
  \item[First] \hfill \\
  The first item
  \item[Second] \hfill \\
  The second item
  \item[Third] \hfill \\
  The third etc \ldots
\end{description}

\section{Quoting}

\begin{quotation}

Quoted text can include other text styles.

\Large But headings shouldn’t use default commands.

\end{quotation}

\section{Comments}

\section{Symbol Escaping}

No-break space (Unicode U+00A0) will be converted to tilde symbol for TeX.

\XeTeX{} doesn't need \TeX{} ligatures for quotation marks, em-dash and other
typographic symbols—they can be ‘used’ “directly”. Euro symbol can be used for
as less as €10. Ellipsis… in the text should be a Unicode character.

Should \TeX{} ligatures be ``used''—they will be printed <<as is>>.

\TeX{} special symbols are automatically escaped: \%, \$, \{, \}, \_, \#,
\&, \textbackslash, \textasciitilde, \textasciicircum.

\section{Other Features}

Document title and author should be defined in a wrapping \TeX{} file, as well
as sepecial markup for abstract, placement of table of contents and
bibliography. Default export intends to produce good quality document with
minimal set of features. Another approach would be to use custom markup in the
document (i.e. labels) and process it during the export stage.

\bibliographystyle{plain}
\bibliography{basic}

\begin{thebibliography}{9}

\bibitem{lamport94} Leslie Lamport, \emph{\LaTeX: A Document Preparation System}.
Addison Wesley, Massachusetts, 2nd Edition, 1994.

\end{thebibliography}

\hrulefill

\section{Not Possible or Incompatible}

\begin{itemize}

\item Alternative headings for sections (to be used in TOC).

\item Explicit hyphenation.

\item Prohibiting ligatures by \{\} (but we might use special Unicode symbols
for that).

\item \TeX{} ligatures for typographic symbols (e.g. ---) are not processed
by default.

\item Difference between breakable (\texttt{\textbackslash{}slash}) and
non-breakable slash mark.

\end{itemize}

\end{document}
